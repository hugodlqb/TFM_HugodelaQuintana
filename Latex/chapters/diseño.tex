\chapter{Diseño del proyecto}

En este capítulo se describe el diseño del proyecto

\section{Dimensionamiento del proyecto}

explicaren detalle que haran los componentes, con que equipos se cuentan (rollo suelo radiante, fancoil...)

\section{Seccion (buscar nombre en gira app)}

rollo seccion iluminacion, seccion alarmas, perianas, clima..

\section{Funcionalidad}

\begin{itemize}
\item \textbf{Iluminación: }en esta sección se agrupan todos los elementos de la vivienda que comparten el mismo desempeño: iluminar, independientemente de si se tratan de luminarias del tipo ON/OFF o del tipo regulable. Cada punto de luz se controlará independiente del resto de puntos, y se podrá hacer desde uno o más de los pulsadores instalados, desde la pantalla del G1 o desde la aplicación del móvil. Se han habilitado mediante programación los llamados servicios centralizados, permitiendo así controlar conjuntos de luminarias como por ejemplo, el centralizado general, que permite apagar todas las luces de la vivienda, permitiendo así al cliente poder salir de la vivienda con la seguridad de no estar malgastando energía, ahorrando de esta manera en su factura eléctrica. Para esta sección también se ha hecho uso de los sistemas sensoriales de movimiento, presencia y luminosidad: la luminaria del trastero, las de los pasillos y el hall se encienden de manera automática al detectar movimiento en ellos; en el salón, sin embargo, se ha utilizado un detector de presencia al no tratarse de una zona de paso, que en combinación con la cuantificación de luminosidad en la estancia, enciende de manera automática la lámpara si detecta a alguna persona y la regula en función del valor de luminosidad sensado.
\item \textbf{Recuperador de CO\textsubscript{2}: }esta sección contará con un detector de partículas de CO\textsubscript{2}, que activará la señal para dar la orden a un sistema de extracción y renovación de aire mediante un sistema de ventiladores. De manera habitual, el sistema de recuperación debe permanecer en funcionamiento con el menor nivel de ventilación activado, y deberá ser apagado y reactivado por el cliente de manera manual mediante un switch habilitado tanto en la G1 como en la aplicación del móvil. Si el sistema se encuentra operativo, el sensor lanzará señales de activación de los diferentes niveles de velocidad de los ventiladores en función de la cantidad de partículas detectadas. Concretamente, se han programado tres límites: el primero de ellos, cuando se detectan entre 500 y 1000 partículas de CO\textsubscript{2} por millón analizadas, un segundo limite que activa el siguiente nivel de velocidad de los ventiladores cuando detecta entre 1000 y 1500 partículas por millón, y finalmente el tercer limite, que activa la máxima velocidad al detectar una cantidad superior a las 1500 partículas por millón. Por añadidura, al activar cualquier luz de alguno de los baños, el ventilador entrará en velocidad máxima durante 10 minutos, independientemente de los valores sensados, al igual que al accionar una de las teclas de los pulsadores, que ha sido programada para lanzar esta función de recuperación.
\item \textbf{Seguridad ante incendios: }esta sección contará con detectores de humo instalados en diferentes habitaciones de la vivienda, que una vez activados, enviarán una señal de activación a la sirena de alarma y cortarán el suministro de gas mediante el cierre de la electroválvula. Esta alarma enviará una notificación tipo Push a los dispositivos móviles conectados con la instalación, permitiendo la desactivación de la señal acústica, manteniendo el cierre de la electroválvula.
\item \textbf{Seguridad ante intrusiones: }para evitar en la medida de lo posible la irrupción de personas no deseadas en la vivienda, esta sección hace uso de los sensores de movimiento y presencia utilizados en la sección de iluminación, para lanzar la señal de alarma que activa la sirena y envía un mensaje Push a los dispositivos móviles conectados con la aplicación.
\item \textbf{Seguridad ante inundaciones: }esta sección tendrá un funcionamiento similar a la referida a seguridad frente incendios, únicamente cambiarán los sensores de humo por otros de inundación, que estarán ubicados en las zonas húmedas de la casa, como son los baños y la cocina.
\item \textbf{Clima:} esta sección será la encargada de controlar la temperatura de la vivienda haciendo uso de diversos elementos. Entre ellos encontramos los termostatos, que harán las veces de interfaz con el usuario en cada habitación gracias a sus pulsadores y displays, permitiendo controlar la velocidad de los ventiladores en caso de que se encuentren en funcionamiento, controlar si se desea o no aclimatar esa estancia y que temperatura es la requerida por el cliente. Internamente, también se encargará de gestionar al resto de equipos, indicando cuando y como deben encenderse y actuar en función de la temperatura sensada y la temperatura de consigna deseada. \\
Durante el verano, el equipo de refrigeración se basará únicamente en el uso del equipo de fancoil, evitando así la aparición de hongos y humedades producidos por la condensación proveniente del uso del suelo radiante. El usuario podrá elegir entre dos modalidades de actuación: la manual, en la que podrá seleccionar entre tres niveles la velocidad a la que desea que se expulse el aire refrigerado en el equipo fancoil, o bien el modo automático, en el que el módulo de actuación de las rejillas seleccionará de entre las tres velocidades mencionadas anteriormente una de ellas en función de un factor de ponderación. Este factor, limitado a un máximo de 100 puntos, será implementado durante su programación, dotando de un valor numérico a cada habitación en función de diversos parámetros como pudiera ser su tamaño o el nivel de confort que requiera. Un ejemplo claro de esto pudiera ser la actuación de uno de los equipos instalados que se encargaría de controlar la temperatura de la cocina, el salón y una de las habitaciones, a los que se les ha otorgado una ponderación de 15 puntos, 55 puntos y 30 puntos respectivamente. Si en una primera situación, hubiese demanda de la cocina y la habitación, al sumar 45 puntos no llegarían al requerimiento del segundo nivel de velocidad, activando así el primer nivel; mientras que si se activasen cocina y salón simultáneamente, al superar el baremo, sí que entraría en funcionamiento la segunda velocidad del equipo.
\begin{table}[t]
\begin{center}
\begin{tabular}{| r | l | c |}
\textbf{Valor ponderado de demanda} & \textbf{Nivel activo} \\ \hline
0 & OFF \\\\ \hline
1-33 & Velocidad 1 \\ \hline
34-66 & Velocidad 2 \\ \hline
67-100 & Velocidad 3 \\ \hline
\end{tabular}
\caption{Ponderación velocidad fancoils}
\label{tab:vel_fancoils}
\end{center}
\end{table}
En cambio, durante el modo invierno el equipo principal que actuará será el de suelo radiante. En este caso el termostato se encargará de enviar una orden de apertura a las válvulas de las habitaciones en las que la temperatura se encuentra por debajo de la demanda, y una señal de arranque a la caldera en cuanto que una de estas válvulas es abierta, haciendo circular el agua caliente a través del entramado de tuberías instaladas en el suelo. En el momento que la temperatura de la habitación y la de consigna tienen una diferencia mayor de 3ºC, se ha programado el comienzo de actuación del sistema secundario: el sistema de fancoil. El sistema de aerotermia tendrá exactamente el mismo modo de funcionamiento que en el modo verano, activando sus velocidades en función de la ponderación de las habitaciones que se encuentren en demanda al encontrarse funcionando en modo automático, o pudiendo ser elegida por el usuario en el modo manual.
\item \textbf{Consumos:} por petición del cliente, se debe hacer un seguimiento de los consumos tanto de luz, como de agua y gas que se dan en la vivienda, por lo que se han habilitado lectores adicionales a los respectivos instalados por las compañías de suministro, permitiendo visualizar en la pantalla del G1 o en la aplicación del móvil el consumo instantáneo de tensión, corriente, potencia, agua o gas; el consumo total de energía, de agua y gas durante diversos periodos, como por ejemplo el consumo del mes anterior, el consumo desde el día 1 del mes en el que se encuentren o incluso en un periodo definido por el propio usuario.
\end{itemize} 

\section{Ubicación}

Ubi

\section{Conexionado}

\begin{itemize}
\item \textbf{Cuadro eléctrico de domótica } \\ \\
Siguiendo las normas de seguridad recogidas en la normativa \cite{Reglamento:2021}, una de las líneas del sistema trifásico y el neutro de la instalación eléctrica del edificio se hacen pasar en primera instancia por un Interruptor General Automático (IGA), ya que el Interruptor de Control de Potencia (ICP), encargado de cortar el suministro en situaciones de sobrecarga, cortocircuito y en los que la demanda de potencia supera a la potencia contratada, se encuentra integrada en el contador instalado en la vivienda por la compañía de suministro eléctrico. El IGA tendrá como misión principal proteger el resto del circuito en el caso de que se produzca un cortocircuito o se supere la potencia máxima que es capaz de soportar la instalación, como por ejemplo, cuando son conectados demasiados electrodomésticos a la vez. Esta interrupción de la corriente nada tendrá que ver con cuestiones económicas o limitantes en función de lo que se tenga contratado y se pague a la compañía de suministro eléctrico, si no que será limitante en cuanto a las características físicas de la propia instalación, no pudiendo ser mejorada si no son sustituyendo y mejorado alguno de los elementos que la componen. A continuación, se ha instalado un Protector Contra Sobretensiones (PCS), que tal y como indica su nombre será el encargado de proteger el resto de circuitos en las ocasiones en las que se produzcan picos elevados de tensión no controlados, como puede ser el caso del impacto de un rayo, desviando la corriente hacia la toma de tierra, evitando daños en los equipos conectados, en la propia instalación o incluso sobre los usuarios que se encuentran en el interior de la vivienda.\\
Siguiendo el cableado, el siguiente elemento que nos encontramos es el Interruptor Diferencial (ID). Este elemento desarrollará la función de proteger a los usuarios de las fugas de corrientes a tierra que pudiesen producirse por daños o malas conexiones de los electrodomésticos con la instalación eléctrica. En cada vivienda es usual instalar entre dos y tres ID que agrupen varios sistemas con diferentes funcionalidades, facilitando así localizar que la fuga de corriente se está produciendo en alguno de los elementos que a ella se encuentra conectado, pero debido a demanda del cliente, se ha seguido el modelo habitual de instalación aplicado en los cuadros de las viviendas de Alemania, en el que existe un ID por cada una de las funcionalidades que se desarrollan en la instalación, teniendo el sistema de iluminación su propio ID, por ejemplo. \\
En último lugar, y antes de comenzar conectar los elementos, se colocan los últimos elementos de protección del sistema: los Pequeños Interruptores de Potencia (PIA), o como son conocidos comúnmente, Interruptores Automáticos. Estos interruptores sí que es habitual encontrarse uno por cada grupo de elementos con la misma funcionalidad, y tendrán como misión detectar el exceso de consumo en estos grupos, desconectándose de manera automática en tal caso. También son muy útiles en el caso de querer realizar alguna modificación en un sistema concreto, ya que si es preciso desconectarlo, no afectará al resto, que podrán seguir operando de manera normal.
\end{itemize} \newpage
Una vez explicado el conexionado hasta el cuadro de domótica, se entrara en detalle el cableado de este hasta los módulos y mecanismos que componen la instalación domótica que controla la vivienda. Entre ambas partes, se han utilizado dos tipos distintos de bornas para facilitar el peinado de los cables, su distribución y organización a lo largo de los tubos y debido a que la ley vigente dictamina que no es legal ni seguro la conexión directa de cables, teniendo que realizarse está a través de algún elemento de paso y sujeción: las bornas de paso y las de distribución. Para una mayor claridad, las bornas de distribución vendrán representadas en los esquemas con un color verde, y serán utilizadas, tal y como su nombre indica, para distribuir las tensiones que llegan a los PIAs. Estas bornas cuentan con cuatro puertos interconectados entre ellos, ofreciendo así el mismo valor de tensión en cada una de sus salidas y son independientes unas bornas a otras, a menos que se haga una conexión directa entre alguno de sus puertos. \\
Por otro lado, las bornas de paso, representadas en color amarillo, también cuentan con cuatro puertos, pero en esta ocasión no se encuentran conectados entre ellos, si no que tienen una distribución distinta, tal y como se muestra en la imagen. En estas bornas, el segundo y cuarto puerto si cuentan con una conexión interna para así ofrecer la misma caída de tensión en ambos puntos, mientras que el primero y el tercero serán independientes. Estas salidas cuentan con conexiones laterales que les permiten conectarse a la borna contigua al encontrarse enganchadas físicamente unas a otras, sin necesidad de realizar esta conexión mediante cables. Por lo tanto, una fila de bornas de paso compartirá la misma tensión con el resto en su primer y tercer puerto, que serán los asignados a la toma de tierra y al neutro común, respectivamente. Los otros dos puertos, serán independientes del resto de bornas e irán conectadas a la fase, uno de ellos a la entrada desde el actuador y el otro al elemento que se desea controlar.

\begin{itemize}
\item \textbf{Actuadores reguladores y binarios y de persianas} \\ \\
Una vez explicado el conexionado hasta el módulo, se pasa a detallar el cableado desde este hasta el elemento a controlar. Para esta sección, se ha dividido los elementos en tres bloques en función de su conexión: en el primer bloque se encuentran todos los elementos de tipo On/Off conectados al actuador de 24 salidas, al que llegará la fase a una de sus salidas desde la borna de distribución, saliendo por el otro terminal de esa misma salida hacia la borna de paso. La conexión de estos elementos será muy simple: de la borna de paso alimentada con fase saldrá uno de los cables que irán al elemento en cuestión, volviendo desde el otro terminal al puerto del neutro de la borna. Para el segundo tipo de elementos el conexionado será idéntico a los del primer tipo, pero en esta ocasión el actuador no solo permitirá abrir o cerrar el circuito, si no que permitirá regular la corriente de salida, realizando así la función de dimmer. Por ultimo tenemos los elementos tipo ventana, que necesitaran de dos de las salidas del actuador de 24 salidas, que vendrán alimentadas desde la borna de distribución, y saldrán hacia la borna de paso. En esta ocasión, de la borna de paso será necesario sacar tres cables hacia el elemento: uno de ellos llevara el neutro, mientras que los otros dos irán conectados a los terminales que determinan si el motor de la persiana o ventana debe actuar en una dirección o su contraria. Todas estas órdenes serán transmitidas a través del cable de bus KNX al actuar sobre los pulsadores, la pantalla o la aplicación móvil.
\end{itemize}
\begin{flushleft}
\begin{figure}[h]
\includegraphics[width=1.15\textwidth]{figures/conex_ilu.png}   
\caption{Conexionado actuadores reguladores y binarios y de persianas}
\label{fig:conex_ilu}
\end{figure}
\end{flushleft}

\begin{itemize}
\item \textbf{Contadores de consumo} \\ \\
Para los módulos de conteo de consumo de agua y gas será necesario contar con las cajas de registro instaladas en la vivienda por la empresa suministradora, ya que este módulo no medirá los caudales de manera directa. Las cajas de registro se encuentran conectadas tanto a la caldera, para el conteo del caudal de gas, como a todas las tomas de agua de la vivienda, e irán emitiendo pulsos SO que serán  cuantificados a través de un optoacoplador y transformados en un número entero a través de un factor de conversión programado en el dispositivo. \\
Por otro lado, los módulos de consumo eléctrico si realizarán el conteo de manera directa a través de los acopladores de corriente instalados en la fase de entrada de cada uno de los circuitos eléctricos que conforman la carga. Estos acopladores realizarán las mediciones mediante el uso del efecto Hall dado en los cables sobre los que se encuentran “abrazados”, por lo que será necesario planificar un espacio en los tubos de cableado que se encuentran en las paredes de la vivienda.
\end{itemize}
\begin{flushleft}
\begin{figure}[h]
\includegraphics[width=1.15\textwidth]{figures/conex_consumo.png}   
\caption{Conexionado módulos de medidas de consumo}
\label{fig:conex_consumo}
\end{figure}
\end{flushleft}

\begin{itemize}
\item \textbf{} \\ \\
\end{itemize} 