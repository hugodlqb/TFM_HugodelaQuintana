%chapter introduce un nuevo capítulo

\chapter{Resumen}

El presente trabajo fue llevado a cabo con la supervisión y la financiación de la empresa Freedom Ingeniería y Domótica, ubicada en la ciudad de Madrid, España, enmarcada en el sector de las instalaciones eléctricas, las telecomunicaciones y el diseño de sistemas domotizados. \\
Con el fin de lograr el mejor resultado en este proyecto, la empresa pondrá a disposición del autor todos los recursos necesarios para su correcto desarrollo, siempre dentro de los marcos económicos y temporales posibles de otorgar a un proyecto de estas características.\\
 El objetivo del TFM es planificar y diseñar una instalación domótica real en una residencia, así como realizar su puesta en marcha. En cuanto a las funcionalidades a implementar, se considera el control del sistema de iluminación y persianas, implementación y programación de pantallas, control de consumos, control sensorial, fusión de sistemas para el control de la climatización, protocolos de comunicación local y remota. Otros objetivos serían optimizar el diseño de la instalación para ajustarse al máximo rendimiento y ahorro energético, y establecer un sistema de contadores para realizar un sistema de control de consumo de recursos (energía, agua). \\\\
Para lograr alcanzar esos objetivos de manera óptima, se plantearán diversas fases del proyecto en el que estarán marcados una serie de hitos para facilitar así el control de la evolución del mismo, como pueden ser la consecución del software y material necesarios o partes del desarrollo de las programaciones de los mecanismos a implementar.\\\\
Todo ello se deberá definir y concretar con un cliente y unos proveedores reales, atendiendo a sus peticiones y requisitos, por lo que también serán necesarias aptitudes comunicativas y de negociación a la hora de lograr llegar a acuerdos en los puntos en los que sea necesaria su intervención.

\paragraph{Keywords:} tecnología, control, domótica.


\chapter{Abstract}

The present work was carried out with the supervision and financing of Freedom Ingeniería y Domótica, a company based in Madrid, Spain, dedicated to electrical installations, telecommunications and the design of home automation systems. \\
As a means to achieve the best result, the company has provided the author with all necessary resources for the development of this project, within the usual budgetary and temporary frames for a design of these characteristics.\\
The purpose of the thesis is to design, plan and implement a real home automation installation. Regarding the functionality requirements, it features the lightning system control, blind controls, consumption control, sensory control, a merging of systems for climate control, and also local and remote communication protocols as well as the implementation and programming of screens. Other objectives would be to optimize the design of the installation so as to get the highest performance and energy saving; and to set up a meter system to be able to control the consumption of resources (energy, water). \\\\
In order to achieve these objectives, the project is conducted in different phases. Milestones have been set in each phase to ease the control of the evolution of the project, such as the acquisition of the needed software, or the different programming stages of the implemented mechanisms. \\\\
Also, all of that must be defined along with the real customer and suppliers, according to their requests and requirements. Therefore, communication and negotiation skills will also be necessary to reach agreements on the points where their intervention is necessary.


\paragraph{Keywords:} technology, control, home automation.