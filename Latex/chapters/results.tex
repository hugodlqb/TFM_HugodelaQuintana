\chapter{Pruebas y resultados}

En este capítulo del Trabajo se realizará un análisis de la fase final del proyecto, en la que, una vez finalizada la programación y la instalación de los dispositivos en la vivienda, se hará un examen exhaustivo al comportamiento de todos los mecanismos y la relación entre ellos,  comprobando la robustez del diseño y subsanando pequeños bugs o fallos no identificados en las fases anteriores, así como el funcionamiento adecuado de los equipos y mecanismos que lo componen.

\section{Puesta en marcha}
En esta sección se describirá en detalle la fase final del proyecto, consistente en que, una vez finalizada la instalación física de los componentes en la vivienda, se realizarán una serie de pruebas y comprobaciones para confirmar que tanto la programación como su volcado en los mecanismos han sido correctos, sin llegar a suceder o sin alcanzar algún estado inesperado o incorrecto que hiciese detenerse o incluso fallar el sistema completo. Con estas pruebas también se detectan fallos cometidos durante la fase de instalación del sistema eléctrico como cables mal etiquetados que dan lugar a confusiones y comportamientos anómalos de los mecanismos, o incluso derivaciones y otra serie de cuestiones relacionadas con el cableado y su acometida eléctrica. Todas estas pruebas, chequeos y correcciones se engloban en el término Puesta en Marcha.\\\\

Por motivos de tiempo, las primeras pruebas que se han realizado sobre los equipos son las relacionadas con los módulos lógicos implementados en el X1. Esto es debido a que en caso de no comportarse de manera idéntica en la instalación que en las simulaciones y pruebas en el ordenador, ya sea por contener programación incorrecta o por la influencia de otras variables del sistema no contempladas durante estas simulaciones, su modificación y posterior fase de pruebas y simulaciones tendría un gran coste en términos de tiempo, lo que finalmente se traduce en una pérdida del beneficio y del margen de ganancia que obtendría la empresa, alejándolo por tanto de su objetivo final.\\\\

Debido a este requerimiento de enviar a una persona a realizar pruebas sobre la propia instalación real, las simulaciones y pruebas en el ordenador toman una gran relevancia a la hora de no invertir o gastar recursos y tiempo que podrían restar valor final al total del proyecto. Por lo tanto, utilizando el software \textit{Gira Project Assistant}, se han testeado y comprobado encarecidamente las programaciones lógicas desarrolladas antes de realizar la prueba definitiva en la vivienda.\\\\

Una vez finalizada esta parte de la Puesta en Marcha, se continuará volcando la programación completa a todos los módulos y mecanismos que componen el sistema, permitiendo así realizar el resto de comprobaciones. Estas rutinas de chequeo han sido diseñadas y plasmadas en un documento (confidencial) por el propio ingeniero que ha diseñado el proyecto y posteriormente, ampliadas por expertos con mayor experiencia y conciencia de los fallos más habituales y también los más perjudiciales. Esta combinación permite buscar y encontrar con una mayor precisión y rapidez los errores cometidos a cualquier persona con cualificación y conocimientos en la materia, logrando así no copar el tiempo al completo del diseñador y programador. \\

Cuando todos los mecanismos hayan recibido su programación, se comenzará realizando las pruebas más simples, rápidas y sencillas como el ON/OFF de la las luminarias o la subida y bajada de las persianas, e irá incrementando su complejidad hasta finalmente realizar un chequeo completo del sistema de climatización teniendo en cuenta todas las posibilidades de combinaciones y comportamientos posibles que este sistema puede llegar a alcanzar. \\\\
Una vez se han validado todas las pruebas y especificaciones recogidas en el documento anteriormente mencionado, se realizará una demostración al cliente, explicando a su vez el funcionamiento final que tendrá el sistema instalado en su vivienda, validando de esta manera la consecución de los requisitos iniciales planteados y dando por finalizando así la Puesta en Marcha, y por tanto, el proyecto.


\section{Resultados}

Una vez expuestos las pruebas en la sección anterior, aquí se deben comentar y analizar su validez.

\section{Mejoras}

Una vez expuestos las pruebas en la sección anterior, aquí se deben comentar y analizar su validez.