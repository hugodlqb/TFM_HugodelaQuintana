\chapter{Conclusiones}

\section{Conclusiones}

Tras la realización completa de este proyecto, pasando por todas las fases de su desarrollo: desde su diseño hasta su implementación y puesta en marcha, pasando por el estudio de los materiales y componentes a instalar, se han obtenido una serie de conclusiones que a continuación se pasan a detallar. \\\\
En primer lugar, se ha determinado que una buena organización permite el desarrollo óptimo del proceso, tanto en la cuestión temporal como económica, ya que al encontrarse planificada cada una de estas fases y conociendo los materiales que van a ser necesarios en cada una de ellas, se podrá disponer del personal necesario en cada momento para la consecución de los objetivos marcados, sin ser penalizados con cualquier desavenencia causada por estos mismos motivos. En este aspecto también se destacan las facilidades que otorga el ser poseedor de los contactos necesarios de los proveedores de material, para lograr los precios más bajos y las menores demoras posibles en la entrega del material.\\
Por otro lado, también favorece el desarrollo adecuado el conocer de primera mano los requisitos exigidos para la creación del sistema, así como las peculiaridades que se van a encontrar a la hora de su programación e instalación en la vivienda, como puede ser el espacio y otras capacidades físicas de la vivienda.\\
Otra conclusión obtenida del desarrollo de este proyecto es que, como se pude observar en el caso de la programación de las luminarias tipo Dimmer, es que las simulaciones realizadas en un ordenador siempre funcionarán de manera correcta, pero esto no implica que al volcar esa programación en los módulos reales, el sistema vaya a comportarse de la misma manera que lo hacía en las pruebas virtuales, por lo que se ha de tener muy presente y dar una gran importancia a la fase de pruebas y puesta en marcha.\\\\
Finalmente, el cliente quedo satisfecho con el resultado de la instalación, por lo que como conclusión final destacaría la importancia de contar con un equipo cualificado y con experiencia en el sector que se encargue rápidamente de cualquier problema o desavenencia que pueda aparecer con el proceso, así como la fluidez en la comunicación con el propio cliente.


\section{Futuras líneas de trabajo y aplicación}

Como se ha venido comentando a lo largo de todo el documento, esta tecnología se encuentra ahora mismo en plena ola de desarrollo y mejora, por lo que en los próximos años es muy probable que se desarrollen y se saquen al mercado módulos con funcionalidades totalmente innovadoras que permitan  llevar la domótica y la inmótica al siguiente nivel: la domotización de una ciudad completa. Esta futurística idea, podría permitir controlar y gestionar una ciudad entera desde un ordenador, pudiendo llegar a lograr reducir y optimizar el uso de la energía en una amplia región densamente poblada a la vez que se realizan otro tipo de gestiones para mejorar el confort y la vida en la ciudad, como por ejemplo, la gestión de atascos, el control de la iluminación pública o la emisión de gases contaminantes, entre muchos otros. Algo más cercano, y que ya comienza a darse en la realidad, es la domotización de los grandes edificios de nueva construcción, acercándose así a la idea del control de todos los edificios de una región  y aplicando notables mejoras en la eficiencia energética y el consumo. \\\\
Por otro lado, y con los nuevos avances tecnológicos, es bastante probable que la domótica alcance al sector de la Salud, y comiencen a desarrollarse mecanismos y módulos que permitan a los pacientes de hospitales o personas con necesidades especiales tener una vida más cómoda y  sencilla en lo que respecta a la realización de tareas no complejas que se realizan a diario, como su desplazamiento por interiores, hacer la compra o darse una ducha.\\\\
Concretamente para este proyecto, una de las futuras líneas de trabajo que se generan, es aquella que se comentó al comienzo del documento, que hablaba acerca de la modularidad del sistema y la capacidad de que la vivienda pueda ser dividida en dos en un futuro. Esta mejora precisaría de poco trabajo, ya que, con esta perspectiva, el ingeniero decidió preparar todo el sistema para que esta transición fuese lo menos laboriosa posible, aprovechando el tener fresco todo el sistema y el conocimiento absoluto de su funcionamiento, dos factores importantes de cara a que esta modificación sea realizada en un espacio de tiempo largo o bien, sea otra persona completamente diferente el que se encargue de implementarla. De este aspecto también se obtiene como conclusión que, haciendo uso de una buena documentación y el uso de nombres genéricos y muy concisos, cualquier persona puede integrarse de manera relativamente sencilla en el desarrollo o mejora de esta instalación.
