\chapter{Estado del Arte}

Con muy pocos años de rodaje, la domótica se ha convertido en uno de los sectores con mayor relevancia y perspectiva de incremento y potenciación futura, gracias a los avances tecnológicos recogidos en numerosos artículos y publicaciones, como el caso de “Domótica: Edificios Inteligentes” \cite{EI:2004}. Estos avances que van de la mano de otros sectores como por ejemplo, el social con su continua lucha hacia la inclusión total de la población o el diseño de interiores y su afán por convertir nuestros hogares en una prolongación más de nuestro cuerpo, para hacernos sentir lo más cómodos y confortables posible. Pero es, sin duda, la apuesta de la sociedad por la inclusión de esta tecnología en el sector de la construcción lo que provoca su gran avance, mejorando e incluso desarrollando nuevas funcionalidades, lenguajes de programación y protocolos de comunicación.\\\\
Esta evolución ha ido permitiendo de manera progresiva que, poco a poco los sistemas domóticos integren una mayor cantidad de mecanismos, ampliando así el catálogo de funcionalidades que pueden ofrecer, a la par que estos son diseñados para actuar de manera más independiente y autónoma. En un principio, los sistemas domóticos eran centralizados, es decir, todos los mecanismos y sus parámetros eran controlados desde una central; en cambio, a día de hoy, los elementos son inteligentes y pueden portar su propia programación, y actuar en consonancia a ella, convirtiéndose en sistemas descentralizados. Esta cualidad, obtenida con el paso del tiempo y la integración de los avances tecnológicos, ha permitido la mejora de 4 áreas fundamentales en el control de una vivienda: la seguridad, la gestión energética, el confort y las comunicaciones  \cite{Andes:2011}, como se demostró en la Universidad Católica del Ecuador \cite{Ecuador:2017}, donde gracias a la mejora de los protocolos de comunicación, en especial, la implementación del X10, se pudo domotizar una escuela al completo, mejorando así la calidad de la enseñanza y la satisfacción de sus alumnos y profesores.\\\\
En el caso del ámbito de la seguridad doméstica, la domótica explota de manera mucho más eficiente que los sistemas tradicionales, en al menos 3 de los campos más demandados, integrándolos con el resto de dispositivos. Uno de estos campos será el enfocado hacia las alarmas técnicas, que mediante sistemas sensoriales, podrán avisar al usuario de que existe un elemento en la casa que se encuentra averiado o realizando un mal funcionamiento, como pueda ser una tubería de agua rota o un horno demasiado caliente que quema la comida. Si el problema cuenta con una mayor gravedad o se requiere controlar espacios de manera más certera, incluso se podrán conectar estas alarmas con los servicios de emergencia o reparación profesionales, como pueden ser los bomberos. \\\\
Haciendo uso de esa misma capacidad de conectar con sistemas externos, encontramos los otros 2 campos: la seguridad de los bienes, que mediante detectores de presencia y diferentes actuadores, como alarmas sonoras o cegadoras, tratara de evitar la entrada o sustracción de bienes de la vivienda, enviando un aviso al cuerpo de policía; y la teleasistencia, muy útil especialmente para personas mayores o enfermas, que podrán conectar con los servicios sanitarios que requieran con la simple pulsación de un botón.\\\\
En cuanto a la gestión energética, la domótica también se ha hecho un hueco en este sector gracias a la integración de elementos como temporizadores, termostatos, relojes temporizadores con otros más habituales, como los contadores de consumo, que conectados a través de actuadores, crean un sistema eficiente de ahorro de energía.\\\\

Desde el comienzo de su aplicación en los años 80, principalmente en EE.UU. y Japón, y su popularización en los 90 en países de Europa como Francia, Alemania u otros países nórdicos, la domótica ha ido de la mano del mercado de la construcción. Este sector, ya desde 2004, impulsaba el crecimiento del uso de la domótica al ser instalado en el 7\% \cite{Ikei:2004} de las nuevas promociones, propiciando la aparición de comités de expertos que comienzan a regular y afianzar el termino domótica, creando en Europa en 2006 la Especificación de AENOR EA2006 \cite{direct:2004}, pionera de este sector, que determinaría los requisitos mínimos que debía dispensar una vivienda para poder ser considerada domótica. Una vez fijados la base del concepto, instituciones  como el Ministerio de Industria y Turismo pronto comenzaron a realizar otro conjunto de guías y manuales, como por ejemplo, la Guía Técnica de Aplicación ITC-BT-51 \cite{BOE:2002} creada en 2007, que sentaba catedra acerca de “los requisitos específicos de la instalación de los sistemas de automatización, gestión técnica de la energía y seguridad para viviendas y edificios, también conocidos como sistemas domóticos”.\\\\ 
El estallido de la burbuja inmobiliaria acompañado a la crisis financiera global de finales de la primera década del siglo provoco una caída del 62\% en la construcción de nuevas viviendas, truncando la tendencia a la alza que tenía el mercado domótico, llegando a traducir su incidencia en un descenso del 60\% de la instalación en vivienda nueva \cite{AED:2011}, lejos de la previsiones vaticinadas por numerosos medios \cite{mundo:2010} \cite{Info:2008}, que predecía un 25\% de viviendas de nueva obra con instalación domótica integrada en el año 2010.\\\\
 La domótica, actualmente, se encuentra en continuo desarrollo y enfrentándose a diversos conflictos o problemas como su alto coste de instalación, la falta de personal cualificado, la normalización de los diversos softwares que existen en los diferentes mercados del mundo, la dependencia del sistema eléctrico o los problemas del mundo de la informática, como pueden ser los hackers \cite{cerda:2018}, dando sensación de inseguridad dentro de la tu propia casa. Por lo tanto, es una ciencia con un potencial muy grande pero debe superar los problemas básicos con que se encuentra cualquier tecnología durante su fase de desarrollo, para poder crecer y convertirse en un referente en cuanto a los servicios que ofrece y lo óptimo de sus resultados.\\\\
A día de hoy, ya se ha implementado en 44 millones de hogares en Europa y Norte América \cite{HT:2014}, brindando así la certeza del potencial con el que cuenta esta tecnología y el crecimiento exponencial en el que se encuentra: en 2015, en Europa existían únicamente 5.3 millones de viviendas frente a los 18 millones de 2020. Esta cifra lejos de congelarse, continua en crecimiento con la meta de alcanzar un 20\% de casas domotizadas del total en Europa para el año 2025 \cite{Berg:2020}. En cuanto a su desarrollo en España, se espera que el auge de la domótica logre alcanzar un 300\% para el año 2024, realizándose en casi un 60\% de instalaciones de nueva construcción \cite{Portal:2020} y encontrándose ya en el 40\% de los hogares en forma de dispositivo inteligente

