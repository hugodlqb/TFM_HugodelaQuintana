\chapter{State of the Art}

Con muy pocos años de rodaje, la domótica se ha convertido en uno de los sectores con mayor relevancia y perspectiva de incremento y potenciación futura, gracias a los avances tecnológicos. Estos avances que van de la mano de otros sectores como por ejemplo, el social con su continua lucha hacia la inclusión total de la población o el diseño de interiores y su afán por convertir nuestros hogares en una prolongación más de nuestro cuerpo, para hacernos sentir lo más cómodos y confortables posible. Pero es, sin duda, la apuesta de la sociedad por la inclusión de esta tecnología en el sector de la construcción lo que provoca su gran avance, mejorando e incluso desarrollando nuevas funcionalidades, lenguajes de programación y protocolos de comunicación.\\\\
Esta evolución ha ido permitiendo de manera progresiva que, poco a poco los sistemas domóticos integren una mayor cantidad de mecanismos, ampliando así el catálogo de funcionalidades que pueden ofrecer, a la par que estos son diseñados para actuar de manera más independiente y autónoma. En un principio, los sistemas domóticos eran centralizados, es decir, todos los mecanismos y sus parámetros eran controlados desde una central; en cambio, a día de hoy, los elementos son inteligentes y pueden portar su propia programación, y actuar en consonancia a ella, convirtiéndose en sistemas descentralizados. Esta cualidad, obtenida con el paso del tiempo y la integración de los avances tecnológicos, ha permitido la mejora de 4 áreas fundamentales en el control de una vivienda: la seguridad, la gestión energética, el confort y las comunicaciones.\\\\
En el caso del ámbito de la seguridad doméstica, la domótica explota de manera mucho más eficiente que los sistemas tradicionales, en al menos 3 de los campos más demandados, integrándolos con el resto de dispositivos. Uno de estos campos será el enfocado hacia las alarmas técnicas, que mediante sistemas sensoriales, podrán avisar al usuario de que existe un elemento en la casa que se encuentra averiado o realizando un mal funcionamiento, como pueda ser una tubería de agua rota o un horno demasiado caliente que quema la comida. Si el problema cuenta con una mayor gravedad o se requiere controlar espacios de manera más certera, incluso se podrán conectar estas alarmas con los servicios de emergencia o reparación profesionales, como pueden ser los bomberos. \\\\
Haciendo uso de esa misma capacidad de conectar con sistemas externos, encontramos los otros 2 campos: la seguridad de los bienes, que mediante detectores de presencia y diferentes actuadores, como alarmas sonoras o cegadoras, tratara de evitar la entrada o sustracción de bienes de la vivienda, enviando un aviso al cuerpo de policía; y la teleasistencia, muy útil especialmente para personas mayores o enfermas, que podrán conectar con los servicios sanitarios que requieran  con la simple pulsación de un botón.\\\\
En cuanto a la gestión energética, la domótica también se ha hecho un hueco en este sector gracias a la integración de elementos como temporizadores, termostatos, relojes temporizadores con otros más habituales, como los contadores de consumo, que conectados a través de actuadores, crean un sistema eficiente de ahorro de energía.\\\\
La domótica, actualmente, se encuentra en continuo desarrollo y enfrentándose a diversos conflictos o problemas como su alto coste de instalación, la falta de personal cualificado, la normalización de los diversos softwares que existen en los  diferentes mercados del mundo, la dependencia del sistema eléctrico o los problemas del mundo de la informática, como pueden ser los hackers, dando sensación de inseguridad dentro de la tu propia casa. Por lo tanto, es una ciencia con un potencial muy grande pero debe superar los problemas básicos con que se encuentra cualquier tecnología durante su fase de desarrollo, para poder crecer y convertirse en un referente en cuanto a los servicios que ofrece y lo óptimo de sus resultados.
