\chapter{Introducción}

Como pequeña introducción, es importante señalar que todo lo que aparezca en la memoria debe ser original. Si aparecen textos de otros libros, artículos o webs, deben ir convenientemente referencidos.

En este capítulo no deben faltar los siguientes apartados:

\section{Motivación}

Motivación o Marco del proyecto, es donde se cuenta cómo surgió la idea del proyecto y se da un breve resumen explicativo.

\section{Objetivos y campos de aplicación}

Es muy importante señalar el objetivo principal del TFG, así como los objetivos secundarios que se estableciaron al principio o han ido surgiendo durante su elaboración.

\section{Estructura del documento}

En esta sección del Trabajo de Fin de Máster se expone una breve descripción de la organización y distribución de los contenidos que alberga este documento, para una mayor facilidad a la hora de familiarizarse con él.\\\\
En primer lugar, el lector se encontrará con las secciones dedicadas a la introducción al Trabajo. En el marco teórico se realiza una introducción al mundo de la domótica, así como un recorrido histórico por el desarrollo de este campo desde sus orígenes en 1966. Se encuentran ya, en este apartado, explicaciones necesarias para comprender de manera global en qué consistirá este documento. \\\\
Dicha visión global se completa en el apartado siguiente, donde a través de una breve descripción del Estado del Arte el lector conocerá el contexto actual en que se enmarca este Trabajo, así como la situación presente de la tecnología utilizada para el desarrollo de este proyecto.\\\\
A continuación, se expone de manera detallada el Diseño del Proyecto; una sección dedicada a las fases previas de su desarrollo. En ella se definen tanto los recursos como las características del sistema que se pretende implementar: los componentes elegidos, su dimensionamiento, sus funcionalidades, conexiones y ubicaciones.\\\\
Esta descripción del diseño general del sistema que se va a implementar, sienta las bases del apartado siguiente, el desarrollo del proyecto. En él se detalla el proceso de creación e integración de la programación de los mecanismos. \\\\
Para el desarrollo del sistema, serán necesarios recursos humanos, materiales y monetarios que se detallan en la Gestión del Proyecto. Dichos recursos son organizados en torno a unas fases y a un plan de trabajo que se concretan cronológicamente a través de un Diagrama de Gantt. Este apartado permitirá comprender la extensión temporal y, por tanto, la dificultad, que entrañan algunas de estas fases.\\\\
El último apartado permite entender y cohesionar todos los anteriores, pues en él se presentan los resultados y las conclusiones obtenidas tras el desarrollo de este Trabajo. También se incluyen las Futuras Líneas de Investigación en el campo de la domótica y de aplicación del proyecto desarrollado.\\\\
Por último, se han añadido anexos con información relevante de cara a futuras modificaciones o posibles réplicas del sistema. Dicha información también permitirá utilizar el proyecto como guía de aprendizaje para toda persona que, aún sin un conocimiento profundo de la materia, quisiera realizar un diseño complejo de una instalación domótica en una vivienda unifamiliar.
