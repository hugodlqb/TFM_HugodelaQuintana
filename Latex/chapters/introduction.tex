\chapter{Introducción}

En este primer capítulo se realizará una breve introducción del grueso del proyecto, dando al lector una idea de cuales han sido los motivos que han llevado al autor a desarrollar este trabajo. También se realizara un estudio de los diferentes campos en los que esta tecnología podrá ser aplicada, así como cuáles son los objetivos que busca alcanzar una vez se encuentre diseñado e instalado el sistema domótico en la vivienda. También se redactará una breve guía de como se ha organizado el documento, para que el lector pueda sentirse cómodo consultando este documento.

\section{Motivación}

El consumo de energía eléctrica en el mundo ha tenido un gran aumento en los últimos años debido a que, principalmente, la sociedad contemporánea ha crecido con la tecnología de manera exponencial. La sobrepoblación y la poca eficiencia en la gestión energética convierten el consumo de energía eléctrica en un factor preocupante, algo que, como cita Camó \cite{Camo:2015} “[…]siendo vital para la sociedad actual, pone de manifiesto la necesidad de reflexionar y actuar en su uso correcto, algo que se requiere empezar desde ya”, comienza a ser un problema real para la sociedad mundial del futuro más cercano.\\\\

En la actualidad, las viviendas, como muchos otros aspectos del día a día, comienzan a adaptarse a las nuevas tecnologías. Esta necesidad se ve acentuada y, por tanto, acelerada, por los nuevos retos que impone la sociedad y el estilo de vida imperante, la cual persigue aumentar el confort y el bienestar de los ciudadanos, en especial de aquellos que requieren de una mayor accesibilidad para poder desarrollar su vida de la manera más cómoda posible. Conseguir esto de manera sostenible pasa por una gestión energética eficiente que esté en consonancia con la evolución de la tecnología. \\\\

Y es que, en una coyuntura económica donde el único objetivo del progreso es lograr más progreso y teniendo como guía ético imperante al consumo, el verdadero motor de cambio surge en cada individuo, en cada hogar, aunque es necesario que este cambio sea a nivel global para que sus efectos puedan ser percibidos. Así, también las instituciones se han hecho eco de que el progreso sólo puede continuar si es sostenible. Muestra de ello son los Objetivos de Desarrollo Sostenible \cite{ONU:2015} fijados por la Unión Europea primero para 2020 y, en lo que nos concierne actualmente, para 2030. En relación a este proyecto, destaca el Objetivo 7 "Energía asequible y no contaminante"; el Objetivo 11, "Ciudades y comunidades sostenibles", y el Objetivo 12, "Producción y consumo responsables". Tanto la domótica como la inmótica deben, sin duda, formar parte de la estrategia para la consecución de dichos objetivos, pues contribuye a la eficiencia energética de edificios residenciales e institucionales al permitir al usuario conocer y controlar fácilmente sus hábitos de consumo. La reducción de la emisión de CO2 y el menor consumo de energía primaria, por citar dos factores que se verían optimizados, serían tan sólo un ejemplo de cómo la domótica y la inmótica podrían contribuir a la consecución de hogares y edificios más energéticamente sostenibles, para lograr un Planeta más ecológicamente habitado.\\\\
Y es que, después de muchos años haciendo un uso de la energía y las instalaciones eléctricas convencionales, se han observado nuevas necesidades enfocadas a la simplificación y automatización de las tareas domésticas, algo que hasta ahora no había sido relevante. Tal es el caso de la reproducción de música o la teleasistencia a personas en situación de dependencia. Y una de las respuestas a estas nuevas necesidades es, sin duda, la domótica. Enfocándonos en este colectivo, las personas en situación de dependencia, la domótica más que un lujo se convierte en una adquisición de derechos acorde al momento y la sociedad en que vivimos. \\\\
Es en este punto donde hago especial hincapié en aquellas personas con diversidad funcional, ya que para ellas se trata de una ayuda e incluso, de una necesidad para poder desenvolverse en el día a día de forma independiente. \\\\
En ocasiones, sin embargo, parece que todavía existe una gran distancia entre la promesa que las tecnologías ofrecen y la realidad de las personas con diversidad funcional. Principalmente, porque podría llegar a favorecer un alto grado de comunicación y un acceso a otros campos que en las circunstancias actuales son inalcanzables, tales como la educación, el empleo o el ocio. Teniendo en cuenta la vocación social de la persona, todo medio que facilite las relaciones sociales a colectivos con mayores dificultades para mantenerlas, enriquece y alimenta una faceta importantísima para el correcto desarrollo del ser humano en sociedad. \\\\
Fundamentalmente, lo que posibilita la Domótica además de todo lo relacionado con la comunicación, es el control del entorno, ya sea este doméstico o laboral. Permite el uso de todos los aparatos electrónicos y sistemas eléctricos mediante la voz o el empleo de mandos de control de manera remota. Pero no sólo esto, sino un sinfín de oportunidades aún no exploradas, tales como la realización de llamadas de emergencia, la activación de la apertura de puertas de paso o incluso la regulación en altura de los sanitarios. Una de las mayores ventajas de la domótica es que esta ayuda puede ser progresiva, con lo que se puede ir adaptando al deterioro gradual de las personas mayores o de algunas formas de diversidad funcional, por poner algún ejemplo. \\\\
Tiene un componente de ciencia ficción que ya es plenamente utilizable, pero que indudablemente tiene que conseguir, en un futuro inmediato, un desarrollo importantísimo dada su potencialidad. \\\\
El edificio domótico responde y se integra a la perfección dentro del concepto de “diseño para todos” \cite{Colo:2015}. Indudablemente, un edificio que goza de esta tecnología sirve tanto para personas con discapacidad como para los que no la tienen, pues facilita la vida y la hace más confortable para todos. Se trata de aprovechar los avances que aporta la ciencia y adaptarlos al entorno doméstico de manera que tanto la persona como la sociedad en su conjunto se beneficien de ello. \\\\


\section{Objetivos y campos de aplicación}

El objetivo principal de este proyecto, enmarcado en un contexto de enfoque profesional y de integración de tecnologías interdisciplinares, será el de satisfacer las necesidades del cliente atendiendo siempre a la elaboración de una serie de estudios técnicos, organizativos y económicos relativos al diseño de los sistemas a implementar, los equipos que serán instalados y su funcionalidad. \\
Por tanto, para determinar las necesidades mencionadas, se contará con la participación de un cliente real, que se mantendrá en el anonimato por decisión propia. Con dicho cliente, se deberá pactar una solución final que cumpla, por una parte con sus expectativas económicas, funcionales e incluso estéticas; y por otra, con la normativa vigente referente a las instalaciones en domicilios particulares y su seguridad.\\ 
Los requisitos formalizados de manera poco definida por el cliente, y para los que será necesario diseñar una solución serán los siguientes:
\begin{itemize}
\item \textbf{Control del sistema de iluminación:} la instalación deberá contar con lámparas tanto de tipo ON/OFF como regulables/dimmers accionadas desde diversos puntos de la vivienda.
\item \textbf{Control de persianas/ventanas:} el usuario podrá controlar la posición y el movimiento de algunas de las ventanas y persianas instaladas en la vivienda.
\item \textbf{Programación del sistema de climatización: }el cliente deberá poder programar diversos parámetros del sistema de clima y ajustarlos a su criterio, como por ejemplo, el ajuste de la temperatura de consigna, la selección de climatización por zonas, velocidad de los ventiladores... Para ello será posible hacer uso de dos tecnologías diferentes: un sistema de suelo radiante y un equipo de tipo fancoil.
\item \textbf{Control de los consumos: }el propietario podrá obtener una lectura en tiempo real y acumulada de los principales dispendios de la vivienda: agua, luz y gas.
\item \textbf{Aplicaciones sensoriales: }la vivienda deberá poseer ciertos comportamientos determinados por diferentes variables sensoriales como la luz externa, la detección de puertas y ventanas abiertas, la presencia de personas o la aparición de elementos dañinos como humo, excesos de CO\textsubscript{2} o inundaciones.
\item \textbf{Sistema de visualización y control: }la instalación deberá tener implementado un sistema de visualización local que permita controlar y monitorizar los parámetros de todos los elementos alojados en la casa, con representación de diferentes mensajes o avisos de alarma.
\item \textbf{Sistema de control remoto: }el usuario podrá acceder al sistema de visualización y control de manera remota, a través de cualquier dispositivo con conexión a internet desde una app.
\item \textbf{Sistema de notificación: }el cliente deberá recibir una notificación de tipo Push en sus dispositivos de control remoto ante la activación de alguna de las alarmas programadas en el sistema sensorial.
\item \textbf{Sistema modular: }el sistema deberá ser desarrollado inicialmente para el control de una vivienda individual, pero con la posibilidad de convertirse en el futuro en dos viviendas independientes la una de la otra.
\end{itemize}
De manera secundaria, los objetivos subsiguientes que se plantean para este designio son los propios de un proyecto profesional de vocación empresarial, de los que podemos destacar, entre otros, el beneficio económico. En concreto, la rentabilidad económica deberá ser tenida en cuenta durante todo el desarrollo del proyecto: desde las fases iniciales en las que ya se necesitará emplear recursos humanos, hasta el último momento, pues influirá en la satisfacción y la valoración del proyecto por parte del cliente. Al tratarse de un objetivo con una meta poco definida, todo ahorro o beneficio hará que, en el balance final, éste se dé por alcanzado y completado con mayor certeza. Para mejorar los resultados en este plano, y una vez definidos los requisitos del alcance de la instalación, se deberá contactar y negociar con distintos proveedores para intentar obtener el mejor rendimiento económico a la hora de comprar los componentes necesarios para la instalación, sin que la obtención del mejor precio suponga una demora adicional en su programación, su instalación y su mantenimiento post-instalación por falta de stock, dificultades de tipo logísticas o baja calidad en los equipos.\\\\
 
Por otro lado, y en cuanto a los campos de aplicación sobre los que sería posible integrar un control domótico o inmótico, es un apunte importante a remarcar el hecho de que, al igual que los sistemas sensoriales, sus posibilidades tienden prácticamente hasta el límite de la imaginación o de las necesidades del cliente utilizando determinada tecnología para llevarlo a cabo, teniendo como techo, al menos en la actualidad, los avances tecnológicos en la rama de las comunicaciones y en la rama de los detectores y sensores electrónicos que se encuentran a la venta hoy en día. \\\\ 
Hoy por hoy, las principales funcionalidades con las que se están dotando esta clase de sistemas son, principalmente: el aumento del confort de los usuarios en sus viviendas domotizadas, así como en los negocios y establecimientos inmotizados; a la par que el aumento del control que éste ejerce sobre el sistema y la información que el sistema le proporciona para facilitar y optimizar su dirección. \\\\
Por todo lo anterior, uno de los sectores sociales que mayor beneficio podría obtener de esta tecnología sería el colectivo de personas en situación de dependencia o con movilidad reducida, pues lograrían una mayor autonomía e independencia al domotizar su vivienda.



\section{Estructura del documento}

En esta sección del Trabajo de Fin de Máster se expone una breve descripción de la organización y distribución de los contenidos que alberga este documento, para una mayor facilidad a la hora de familiarizarse con él.\\\\
En primer lugar, el lector se encontrará con las secciones dedicadas a la introducción al Trabajo. En el marco teórico se realiza una introducción al mundo de la domótica, así como un recorrido histórico por el desarrollo de este campo desde sus orígenes en 1966. Se encuentran ya, en este apartado, explicaciones necesarias para comprender de manera global en qué consistirá este documento. \\\\
Dicha visión global se completa en el apartado siguiente, donde a través de una breve descripción del Estado del Arte el lector conocerá el contexto actual en que se enmarca este Trabajo, así como la situación presente de la tecnología utilizada para el desarrollo de este proyecto.\\\\
A continuación, se expone de manera detallada el Diseño del Proyecto; una sección dedicada a las fases previas de su desarrollo. En ella se definen tanto los recursos como las características del sistema que se pretenden implementar: los componentes elegidos, su dimensionamiento, sus funcionalidades, conexiones y ubicaciones.\\\\
Esta descripción del diseño general del sistema que se va a instalar, sienta las bases del apartado siguiente, el Desarrollo del Proyecto. En él se detalla el proceso de creación e integración de la programación de los mecanismos. \\\\
Para el desarrollo del sistema, serán necesarios recursos humanos, materiales y monetarios que se detallan en la Gestión del Proyecto. Dichos recursos son organizados en torno a unas fases y a un plan de trabajo que se concretan cronológicamente a través de un Diagrama de Gantt. Este apartado permitirá comprender la extensión temporal y, por tanto, la dificultad, que entrañan algunas de estas fases.\\\\
El último apartado permite entender y cohesionar todos los anteriores, pues en él se presentan los resultados y las conclusiones obtenidas tras el desarrollo de este Trabajo. También se incluyen las Futuras Líneas de Investigación en el campo de la domótica y de aplicación del proyecto desarrollado.\\\\
Por último, se han añadido anexos con información relevante de cara a futuras modificaciones o posibles réplicas del sistema. Dicha información también permitirá utilizar el proyecto como guía de aprendizaje para toda persona que, aún sin un conocimiento profundo de la materia, quisiera realizar un diseño complejo de una instalación domótica en una vivienda unifamiliar.
