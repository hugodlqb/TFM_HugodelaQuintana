\chapter{Introducción}

Como pequeña introducción, es importante señalar que todo lo que aparezca en la memoria debe ser original. Si aparecen textos de otros libros, artículos o webs, deben ir convenientemente referencidos.

En este capítulo no deben faltar los siguientes apartados:

\section{Motivación}

Motivación o Marco del proyecto, es donde se cuenta cómo surgió la idea del proyecto y se da un breve resumen explicativo.

\section{Objetivos y campos de aplicación}

El objetivo principal de este proyecto, enmarcado en un contexto de enfoque profesional y de integración de tecnologías interdisciplinares, será el de satisfacer las necesidades del cliente atendiendo siempre a la elaboración de una serie de estudios técnicos, organizativos y económicos relativos al diseño de los sistemas a implementar, los equipos que serán instalados y su funcionalidad. \\
Por tanto, para determinar las necesidades mencionadas, se contará con la participación de un cliente real, que se mantendrá en el anonimato por decisión propia. Con dicho cliente, se deberá pactar una solución final que cumpla, por una parte con sus expectativas económicas, funcionales e incluso estéticas; y por otra, con la normativa vigente referente a las instalaciones en domicilios particulares y su seguridad.\\ 
Los requisitos formalizados de manera poco definida por el cliente, y para los que será necesario diseñar una solución serán los siguientes:
\begin{itemize}
\item \textbf{Control del sistema de iluminación:} la instalación deberá contar con lámparas tanto de tipo ON/OFF como regulables/dimmers accionadas desde diversos puntos de la vivienda.
\item \textbf{Control de persianas/ventanas:} el usuario podrá controlar la posición y el movimiento de algunas de las ventanas y persianas instaladas en la vivienda.
\item \textbf{Programación del sistema de climatización: }el cliente deberá poder programar diversos parámetros del sistema de clima y ajustarlos a su criterio, como por ejemplo, el ajuste de la temperatura de consigna, la selección de climatización por zonas, velocidad de los ventiladores... Para ello será posible hacer uso de dos tecnologías diferentes: un sistema de suelo radiante y un equipo de tipo fancoil.
\item \textbf{Control de los consumos: }el propietario podrá obtener una lectura en tiempo real y acumulada de los principales dispendios de la vivienda: agua, luz y gas.
\item \textbf{Aplicaciones sensoriales: }la vivienda deberá poseer ciertos comportamientos determinados por diferentes variables sensoriales como la luz externa, la detección de puertas y ventanas abiertas, la presencia de personas o la aparición de elementos dañinos como humo, excesos de CO\textsubscript{2} o inundaciones.
\item \textbf{Sistema de visualización y control: }la instalación deberá tener implementado un sistema de visualización local que permita controlar y monitorizar los parámetros de todos los elementos alojados en la casa, con representación de diferentes mensajes o avisos de alarma.
\item \textbf{Sistema de control remoto: }el usuario podrá acceder al sistema de visualización y control de manera remota, a través de cualquier dispositivo con conexión a internet desde una app.
\item \textbf{Sistema de notificación: }el cliente deberá recibir una notificación de tipo Push en sus dispositivos de control remoto ante la activación de alguna de las alarmas programadas en el sistema sensorial.
\item \textbf{Sistema modular: }el sistema deberá ser desarrollado inicialmente para el control de una vivienda individual, pero con la posibilidad de convertirse en el futuro en dos viviendas independientes la una de la otra.
\end{itemize}
De manera secundaria, los objetivos subsiguientes que se plantean para este designio son los propios de un proyecto profesional de vocación empresarial, de los que podemos destacar, entre otros, el beneficio económico. En concreto, la rentabilidad económica deberá ser tenida en cuenta durante todo el desarrollo del proyecto: desde las fases iniciales en las que ya se necesitará emplear recursos humanos, hasta el último momento, pues influirá en la satisfacción y la valoración del proyecto por parte del cliente. Al tratarse de un objetivo con una meta poco definida, todo ahorro o beneficio hará que, en el balance final, éste se dé por alcanzado y completado con mayor certeza. Para mejorar los resultados en este plano, y una vez definidos los requisitos del alcance de la instalación, se deberá contactar y negociar con distintos proveedores para intentar obtener el mejor rendimiento económico a la hora de comprar los componentes necesarios para la instalación, sin que la obtención del mejor precio suponga una demora adicional en su programación, su instalación y su mantenimiento post-instalación por falta de stock, dificultades de tipo logísticas o baja calidad en los equipos.\\\\
 
Por otro lado, y en cuanto a los campos de aplicación sobre los que sería posible integrar un control domótico o inmótico, es un apunte importante a remarcar el hecho de que, al igual que los sistemas sensoriales, sus posibilidades tienden prácticamente hasta el límite de la imaginación o de las necesidades del cliente utilizando determinada tecnología para llevarlo a cabo, teniendo como techo, al menos en la actualidad, los avances tecnológicos en la rama de las comunicaciones y en la rama de los detectores y sensores electrónicos que se encuentran a la venta hoy en día. \\\\ 
Hoy por hoy, las principales funcionalidades con las que se están dotando esta clase de sistemas son, principalmente: el aumento del confort de los usuarios en sus viviendas domotizadas, así como en los negocios y establecimientos inmotizados; a la par que el aumento del control que éste ejerce sobre el sistema y la información que el sistema le proporciona para facilitar y optimizar su dirección. \\\\
Por todo lo anterior, uno de los sectores sociales que mayor beneficio podría obtener de esta tecnología sería el colectivo de personas en situación de dependencia o con movilidad reducida, pues lograrían una mayor autonomía e independencia al domotizar su vivienda.



\section{Estructura del documento}

En esta sección del Trabajo de Fin de Máster se expone una breve descripción de la organización y distribución de los contenidos que alberga este documento, para una mayor facilidad a la hora de familiarizarse con él.\\\\
En primer lugar, el lector se encontrará con las secciones dedicadas a la introducción al Trabajo. En el marco teórico se realiza una introducción al mundo de la domótica, así como un recorrido histórico por el desarrollo de este campo desde sus orígenes en 1966. Se encuentran ya, en este apartado, explicaciones necesarias para comprender de manera global en qué consistirá este documento. \\\\
Dicha visión global se completa en el apartado siguiente, donde a través de una breve descripción del Estado del Arte el lector conocerá el contexto actual en que se enmarca este Trabajo, así como la situación presente de la tecnología utilizada para el desarrollo de este proyecto.\\\\
A continuación, se expone de manera detallada el Diseño del Proyecto; una sección dedicada a las fases previas de su desarrollo. En ella se definen tanto los recursos como las características del sistema que se pretenden implementar: los componentes elegidos, su dimensionamiento, sus funcionalidades, conexiones y ubicaciones.\\\\
Esta descripción del diseño general del sistema que se va a instalar, sienta las bases del apartado siguiente, el Desarrollo del Proyecto. En él se detalla el proceso de creación e integración de la programación de los mecanismos. \\\\
Para el desarrollo del sistema, serán necesarios recursos humanos, materiales y monetarios que se detallan en la Gestión del Proyecto. Dichos recursos son organizados en torno a unas fases y a un plan de trabajo que se concretan cronológicamente a través de un Diagrama de Gantt. Este apartado permitirá comprender la extensión temporal y, por tanto, la dificultad, que entrañan algunas de estas fases.\\\\
El último apartado permite entender y cohesionar todos los anteriores, pues en él se presentan los resultados y las conclusiones obtenidas tras el desarrollo de este Trabajo. También se incluyen las Futuras Líneas de Investigación en el campo de la domótica y de aplicación del proyecto desarrollado.\\\\
Por último, se han añadido anexos con información relevante de cara a futuras modificaciones o posibles réplicas del sistema. Dicha información también permitirá utilizar el proyecto como guía de aprendizaje para toda persona que, aún sin un conocimiento profundo de la materia, quisiera realizar un diseño complejo de una instalación domótica en una vivienda unifamiliar.
