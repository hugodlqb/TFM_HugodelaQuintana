\chapter{Conceptos teóricos}

En este capítulo se describen (brevemente) todos los conceptos necesarios para entender el trabajo. No se trata de copiar el contenido de los libros de texto, si no de hacer un resumen de los conceptos necesarios para facilitar la lectura del documento al lector. Se entiende que el lector de un TFG tiene que tener unos conocimientos mínimos sobre el tema.

\section{¿Qué es la domótica?}

Desde el punto de vista etimológico, el término “domótica” proviene de la unión de las palabras \textit{“domus”}, que en latín significa casa, con la palabra griega \textit{“tica”}, cuya interpretación más acertada es “que trabaja por sí solo”. Por tanto, la palabra domótica podría traducirse como “la casa que trabaja de manera autónoma”. Fue acuñada a finales de los años 60 en Francia (\textit{“domotique”}) con la finalidad de crear nuevas palabras en su idioma para las tecnologías emergentes, en lugar de utilizar préstamos lingüísticos o anglicismos. Pero no fue hasta 1988 cuando apareció por primera vez recogida en un diccionario (“Enciclopedia Larousse”) con la siguiente definición \cite{Larousse:1988}: “Concepto de vivienda que integra automatismos en materia de seguridad, gestión de la energía, comunicaciones, etc”. \\\\
Posteriormente se han recogido descripciones más detalladas y acertadas del propio concepto domótica, como es el siguiente\cite{Quezada:2017}: “dícese de la parte de la tecnología (electrónica e informática) que integra el control y supervisión de los elementos existentes en un edificio de oficinas o de viviendas, garantizado por sistemas que realizan varias funciones y que pueden estar conectados entre sí a redes interiores y exteriores de comunicación. Gracias a ello se obtiene un notable ahorro de energía, una eficaz gestión técnica de la vivienda, una buena comunicación con el exterior y un alto nivel de seguridad”.\\\\
Entrando en una definición más formal y tangible del término, se materializa en una técnica física denominada “Sistema domótico”, que está formado por una serie de módulos y mecanismos conexionados entre sí por una red de comunicación a través de un bus de datos, lo que permite el intercambio de información a través de diferentes protocolos de comunicación. Incluso, se suele incluir algún tipo de interfaz que permita a una persona ser partícipe de la concurrencia de toda esta información.\\\\
Todo esto tendrá como fin la automatización de una vivienda, es decir, que varias de las actividades que hasta ahora estaban destinadas a ser realizadas por las personas que las habitaban, pasan a ser tarea del sistema domótico instalado en ella, reduciendo así la necesidad de vigilarlas y controlarlas de manera directa, y logrando adecuar las condiciones ambientales para ofrecer a los inquilinos las mayores prestaciones de confort y seguridad posibles.\\\\
A lo largo de este Trabajo de Fin de Máster se irán desgranando los diferentes procesos necesarios para que una vivienda convencional evolucione hacia el hogar del futuro: la Casa Domotizada. 

\section{Marco histórico}

Los primeros pasos de la domótica comienzan en 1966, cuando James Sutherland, un ingeniero encargado del diseño del sistema de control de plantas de energía fósil y nuclear, utiliza parte del hardware excedente de uno de los proyectos en los que trabajaba para construir una computadora doméstica. Esta máquina primigenia recibió el nombre de ECHO IV y fue instalada en su propia casa. ECHO IV presentaba diversas funcionalidades, como rotar la antena de televisión instalada en el tejado mediante el uso de una máquina de escribir; procesar texto, o incluso transmitir valores de tiempo real a relojes digitales de tipo BCD, entre otras. Debido a sus desfavorables características tanto de tamaño como de consumo, este invento nunca se llegó a comercializar. Sin embargo, logró captar la atención de numerosos investigadores propiciando así el comienzo de la domótica.\\\\
En la década de 1970, aparecieron los primeros sistemas automáticos de pruebas en edificios públicos y de oficinas de los países más avanzados tecnológicamente por aquel entonces: Alemania, Estados Unidos y Japón. Pero no fue hasta finales de la década siguiente, paralelamente a la evolución de los sistemas informáticos y el desarrollo de los componentes electrónicos, cuando se comenzaron a implementar en domicilios particulares. \\\\
La aparición en 1983 del cableado estructurado facilitó el conexionado de los diversos componentes y redes que componen los sistemas domóticos, lo que propagó su implementación en rascacielos o grandes oficinas comerciales. Esto posibilitó una eficiencia y un ahorro de consumo inédito hasta el momento, propiciando así su auge en el ámbito global. \\\\
Estas instalaciones primitivas comenzaron a programarse informáticamente en Estados Unidos en 1984 mediante el software SAVE. Eran regidas por el protocolo de comunicación X-10 y actuadas por los usuarios por medio de accionadores por control remoto, transmitiendo los datos a través de las líneas de baja tensión. \\\\
De la mano de la popularización de los servicios de tele-asistencia en los años 90 y la revolución que supuso la extensión del uso de internet, la domótica evolucionó hasta los complejos sistemas que podemos encontrar en una vivienda actual común: sistemas en los que se permite un control más amplio y exhaustivo, incluso de manera remota, de numerosos dispositivos tecnológicos vía Wi-Fi, gracias al desarrollo de protocolos de comunicación como ZigBee. \\\\
Actualmente, la domótica se encuentra experimentando un fuerte crecimiento gracias a su precio más accesible, que permite que esta tecnología esté cada día al alcance de más gente. Por una parte, los avances tecnológicos abaratan los costes de instalación y mantenimiento de los componentes mecánicos. Igualmente, se ha facilitado la experiencia del usuario y mejorado la usabilidad del sistema gracias a la aparición de numerosas aplicaciones que nos permiten controlar nuestros hogares desde cualquier lugar del planeta en tiempo real. 


\section{¿Qué es KNX?}

knx, protocolos, teoria clase...)