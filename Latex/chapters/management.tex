\chapter{Gestión del proyecto}

En este capítulo se detallaran aquellos aspectos relacionados con la gestión del proyecto, como la planificación temporal y de objetivos que llevará cada una de las fases que completan el proyecto, la organización que se ha seguido a la hora de desarrollarlas y un desglose con los recursos que han sido utilizados para poder llevarlo a cabo siguiendo la planificación diseñada y tratando de ajustar al máximo el margen de beneficios.

\section{Fases del proyecto}

En el desarrollo del proyecto pueden diferenciarse siete fases. En primer lugar, se distingue una primera fase de obtención de los recursos que permitirán iniciar el proyecto. Por una parte, será necesario contar con el software ETS5 y con Gira Project Assistant \cite{GiraApp:2021}. Asimismo, en esta primera fase deberán conocerse las especificaciones y los requisitos a los que debe atenerse, y en base a ello, la bibliografía que será necesaria para poder iniciar la segunda fase: el análisis teórico.\\\\
Previo al diseño del proyecto, se hace necesario estudiar su viabilidad en base a las especificaciones definidas en la primera fase. De ser factible, se procederá al diseño de la solución y a la presentación de la misma al cliente, donde se expondrán las nuevas características o modificaciones del sistema. Tras el fin de la negociación y llegar a un acuerdo, se elaborará el presupuesto y se hará un pedido para obtener el hardware necesario que haya sido elegido en base a las especificaciones finales.\\\\
Una vez recibido el material necesario, comenzará la fase de implementación en ETS5, con un desarrollo de la arquitectura de la vivienda que va a ser domotizada. En esta fase, se deberán elaborar los grupos de direcciones, así como buscar y descargar las bases de datos de los productos. Posteriormente, tras añadir los módulos de los dispositivos del sistema domótico, la implementación finalizará al  asignarles una dirección física a los mismos.\\\\
En cuanto a la cuarta fase, el análisis del desarrollo, comenzará con la programación de los parámetros de los dispositivos y en linkado de sus objetos de comunicación. Posteriormente, será necesario crear las escenas y programar los módulos lógicos, para finalmente volcar la programación sobre los dispositivos físicos, y proceder a la instalación de los mismos en la vivienda.
Para el desarrollo del sistema de visualización, la quinta fase, será necesario crear el servidor del control remoto, diseñar y crear las pantallas de visualización y volcar la programación sobre los dispositivos físicos.\\\\
Una vez esté todo listo, se procederá a la validación y la puesta en marcha. Comprobando previamente que la instalación eléctrica y el cableado sean correctos, será necesario testear las funcionalidades del sistema. En caso de que fuese necesario, en esta fase se podrán depurar los errores e implementar las mejoras pertinentes.


\section{Metodología}
\subsection{Plan de trabajo}

\begin{itemize}
\item \textbf{T1. Fase de recursos}
	\begin{itemize}
	\item Obtención del software ETS5.
	\item Obtención del software Gira Project Assistant.
	\item Obtención de las especificaciones del proyecto.
	\item Obtención de bibliografía. \\
	\end{itemize} 
\item \textbf{T2. Análisis teórico previo}
	\begin{itemize}
	\item Estudio viabilidad y especificaciones del proyecto.
	\item Diseño de la solución.
	\item Ajustes y mediación con el cliente sobre las nuevas características o modificaciones del sistema.
	\item Elaboración del presupuesto.
	\item Elección y pedido del hardware en base a especificaciones finales. \\
	\end{itemize} 
\item \textbf{T3. Implementación en ETS5}
	\begin{itemize}
	\item Desarrollo de la arquitectura de la vivienda en el proyecto.
	\item Elaboración de los grupos de direcciones.
	\item Búsqueda y descarga de las bases de datos de los productos.
	\item Añadir los módulos de los dispositivos del sistema domótico.
	\item Asignar direcciones físicas a los dispositivos. \\
	\end{itemize} 
\item \textbf{T4. Análisis de desarrollo}
	\begin{itemize}
	\item Programación de parámetros de los dispositivos.
	\item Linkado de los objetos de comunicación de los dispositivos.
	\item Creación de escenas.
	\item Programación de módulos lógicos.
	\item Volcado de programación sobre los dispositivos físicos.
	\item Instalación de los dispositivos físicos en la vivienda. \\
	\end{itemize} 
\item \textbf{T5. Desarrollo sistema visualización}
	\begin{itemize}
	\item Creación servidor control remoto.
	\item Diseño y creación de las pantallas de visualización.
	\item Volcado sobre dispositivos físicos. \\
	\end{itemize} 
\item \textbf{T6. Validación y puesta en marcha del sistema}
	\begin{itemize}
	\item Comprobación instalación eléctrica y cableado.
	\item Testeo funcionalidades del sistema.
	\item Depuración de errores e implementación de mejoras. \\
	\end{itemize} 
\item \textbf{T7. Redacción de la memoria final} \\
\end{itemize} 

\subsection{Diagrama de Gantt}

\begin{table}[H]
\begin{ganttchart}[
canvas/.append style={fill=none, draw=black!5, line width=.75pt},
hgrid style/.style={draw=black!5, line width=1pt},
vgrid={*1{draw=black!5, line width=.75pt}},
x unit=1.2 cm,
y unit title=1 cm,
y unit chart=1 cm,
today label font=\small\bfseries,
title/.style={draw=none, fill=none},
title label font=\bfseries\footnotesize,
title label node/.append style={below=7pt},
include title in canvas=false,
bar label font=\mdseries\small\color{black!70},
bar label node/.append style={left=2cm},
bar/.append style={draw=none, fill=black!63},
bar incomplete/.append style={fill=green},
bar progress label font=\mdseries\footnotesize\color{black!70},
group incomplete/.append style={fill=blue},
group left shift=0,
group right shift=0,
group height=.5,
group peaks tip position=0,
group label node/.append style={left=.6cm},
group progress label font=\bfseries\small,
link/.style={-latex, line width=1.5pt, linkred},
link label font=\scriptsize\bfseries,
link label node/.append style={below left=-2pt and 0pt},
]{1}{10}
\gantttitle{Diagrama de Gantt}{10} \\[grid]
\gantttitle{Abril}{4}
\gantttitle{Mayo}{4}
\gantttitle{Junio}{2}\\
\gantttitle[title label node/.append style={below left=7pt and -3pt}]{Semana:\quad1}{1}
\gantttitlelist{2,...,10}{1} \\
\ganttbar[]{\textbf{T1}}{1}{1} \\
\ganttbar[]{\textbf{T2}}{2}{4} \\
\ganttbar[]{\textbf{T3}}{4}{4} \\
\ganttbar[]{\textbf{T4}}{5}{6} \\
\ganttbar[]{\textbf{T5}}{6}{6} \\
\ganttbar[]{\textbf{T6}}{7}{8} \\
\ganttbar[]{\textbf{T7}}{9}{10}
\end{ganttchart}
\caption{Diagrama de Gantt}
\label{tab:diagrama_gantt}
\end{table}

\newpage
\section{Recursos y Material}

Al tratarse de un proyecto enmarcado en el ámbito profesional, el proyecto contará con el apoyo de la empresa a la que se ha adjudicado la obra, por lo que se contará con profesionales con gran experiencia en el sector para realizar la instalación del sistema eléctrico y domótico en la vivienda. Para realizar esta parte, se contará con dos instaladores, mientras que durante la parte de puesta en marcha y pruebas, se podrá disponer de uno de ellos únicamente en las situaciones específicas que así lo requieran, debiendo ser el propio Ingeniero quien soluciones problemas menores en la instalación, como por ejemplo, reconfigurar el cableado o reajustar los parámetros de los mecanismos. Por otro lado, el material será también adelantado por la empresa para poder comenzar su programación y su puesta en marcha cuanto antes y previo pago del cliente, acelerando y facilitando todo el proceso.

\subsection{Presupuesto}

En esta sección se realizará un recuento del precio y las unidades de los mecanismos adquiridos e instalados en la vivienda. Pese a que, tanto el beneficio económico como el tiempo de programación se vean perjudicados al adquirir los mecanismos a diferentes proveedores \cite{Gira:2021} \cite{Zennio:2021}, es necesario la conjugación de varios de ellos para lograr cumplir con las expectativas del cliente en cuanto al alcance de la instalación. Por tanto, se detallará también la empresa que proporciona los módulos:

\begin{flushleft}
\begin{longtable}[H]{|c|p{4cm}|c|c|c|c|}
\hline 
\rule[0mm]{0mm}{5mm}
\multirow{2}{*}{Referencia} &  \multirow{2}{*}{Descripción} & \multirow{2}{*}{ Fabricante} &  \multirow{2}{*}{Cantidad} & Precio  & Precio \\
&  &  &  &  Unitario &  Total\\
\hline
\hline
\endhead
\rule[0mm]{0mm}{4mm}
 \multirow{2}{*}{205127} & Detect. Movimiento  &  \multirow{2}{*}{Gira} &  \multirow{2}{*}{4} &  \multirow{2}{*}{169,67} &  \multirow{2}{*}{678,68}\\
 &  Komfort 2,2m KNX & & & &\\
\hline
\rule[0mm]{0mm}{4mm}
200800 & Acoplador de bus KNX & Gira & 6 & 51,29 & 307,74\\
\hline
\rule[0mm]{0mm}{4mm}
\multirow{2}{*}{222500} & Detect. Presencia  & \multirow{2}{*}{Gira} & \multirow{2}{*}{1} & \multirow{2}{*}{231,86} & \multirow{2}{*}{231,86}\\
 &  Mini Komfort KNX & & & &\\
\hline
\rule[0mm]{0mm}{4mm}
 \multirow{2}{*}{204127} & Detect. Movimiento &  \multirow{2}{*}{Gira} &  \multirow{2}{*}{1} &  \multirow{2}{*}{118,67} &  \multirow{2}{*}{118,67}\\
 &  Standard 2,2m KNX & & & &\\
\hline
\rule[0mm]{0mm}{4mm}
\rule[0mm]{0mm}{4mm}
\multirow{2}{*}{210427} & Sensor CO2 + & \multirow{2}{*}{Gira} & \multirow{2}{*}{1} & \multirow{2}{*}{336,58} & \multirow{2}{*}{336,58}\\
 &  Humedad KNX & & & &\\
\hline
\rule[0mm]{0mm}{4mm}
 \multirow{2}{*}{18200} & Pulsador KNX de 2 elem. &  \multirow{2}{*}{Gira} &  \multirow{2}{*}{37} &  \multirow{2}{*}{79,22} &  \multirow{2}{*}{2931,14}\\
 &  con mando de 1 punto & & & &\\
\hline
\rule[0mm]{0mm}{4mm}
29527 & Teclas basculantes & Gira & 35 & 4,85 & 169,75\\
\hline
\rule[0mm]{0mm}{4mm}
 \multirow{2}{*}{29427} & Teclas basculantes &  \multirow{2}{*}{Gira} &  \multirow{2}{*}{2} &  \multirow{2}{*}{6,66} &  \multirow{2}{*}{13,32}\\
 &  serigrafiadas con flecha & & & &\\
\hline
\rule[0mm]{0mm}{4mm}
\multirow{2}{*}{213000} & Fuente alimentacion& \multirow{2}{*}{Gira} & \multirow{2}{*}{1} & \multirow{2}{*}{323,37} & \multirow{2}{*}{323,37}\\
 &   640mA KNX  & & & &\\
\hline
\rule[0mm]{0mm}{4mm}
 \multirow{2}{*}{504000} & Actuador de conmutación &  \multirow{2}{*}{Gira} &  \multirow{2}{*}{2} &  \multirow{2}{*}{772,8} &  \multirow{2}{*}{1545,6}\\
 &  24 outs / 12 persianas 16A & & & &\\
\hline
\rule[0mm]{0mm}{4mm}
\multirow{2}{*}{212900} & Actuador calefaccion  & \multirow{2}{*}{Gira} & \multirow{2}{*}{2} & \multirow{2}{*}{231,83} & \multirow{2}{*}{463,66}\\
 & 6 elementos KNX & & & &\\
\hline
\rule[0mm]{0mm}{4mm}
 \multirow{2}{*}{212600} & Entrada binaria KNX de  &  \multirow{2}{*}{Gira} &  \multirow{2}{*}{4} &  \multirow{2}{*}{230,98} &  \multirow{2}{*}{923,92}\\
 & 6 elementos 10-230 V CA/CC & & & &\\
\hline
\rule[0mm]{0mm}{4mm}
 233602& Detector de humos & Gira & 4 & 75 & 300\\
\hline
\rule[0mm]{0mm}{4mm}
\multirow{2}{*}{234300} & Módulo KNX para  & \multirow{2}{*}{Gira} & \multirow{2}{*}{4} & \multirow{2}{*}{115,57} & \multirow{2}{*}{462,28}\\
 & detector de humo & & & &\\
\hline
\rule[0mm]{0mm}{4mm}
209600 & Gira X1 & Gira & 1 & 835,58 & 835,58\\
\hline
\rule[0mm]{0mm}{4mm}
206912 & Gira G1 & Gira & 1 & 1000,61 & 1000,61\\
\hline
\rule[0mm]{0mm}{4mm}
 \multirow{2}{*}{202500} & Actuador de regulacion KNX&  \multirow{2}{*}{Gira} &  \multirow{2}{*}{4} &  \multirow{2}{*}{529,2} &  \multirow{2}{*}{2116,8}\\
 & 4 elementos Komfort & & & &\\
\hline
\rule[0mm]{0mm}{4mm}
 \multirow{2}{*}{KCI 4 S0} & Interfaz KNX para  & \multirow{2}{*}{ Zennio} &  \multirow{2}{*}{2} &  \multirow{2}{*}{83,3} &  \multirow{2}{*}{166,6}\\
 & contadores de consumo & & & &\\
\hline
\rule[0mm]{0mm}{4mm}
\multirow{2}{*}{ZVI-F55D} & Panel táctil capacitivo  & \multirow{2}{*}{Zennio} & \multirow{2}{*}{6} & \multirow{2}{*}{116,2} & \multirow{2}{*}{697,2}\\
 & con display & & & &\\
\hline
\rule[0mm]{0mm}{4mm}
\multirow{2}{*}{ZCL-ZB4} &Actuador de clima con& \multirow{2}{*}{Zennio} &\multirow{2}{*}{2} & \multirow{2}{*}{125,3} & \multirow{2}{*}{250,6}\\
 & zonificación de 4 zonas & & & &\\
\hline
\rule[0mm]{0mm}{4mm}
\multirow{2}{*}{ZIO-KESP} & Medidor de energía & \multirow{2}{*}{Zennio} & \multirow{2}{*}{2} & \multirow{2}{*}{104,3} & \multirow{2}{*}{ 208,6}\\
 & eléctrica KNX KES Plus  & & & &\\
\hline
\rule[0mm]{0mm}{4mm}
{\small ZN1AC-CST120} & Transformador de corriente & Zennio & 6 & 16,8 & 100,8\\
\hline
\hline
\rule[0mm]{0mm}{4mm}
 & & & &TOTAL&12.759,56 €\\
\hline 
\caption{Presupuesto materiales}
\label{tab:tabla_presupuesto_mat}
\end{longtable}
\end{flushleft}

\begin{center}
\begin{longtable}[H]{|c|c|c|c|c|c|}
\hline 
\rule[0mm]{0mm}{5mm}
\multirow{2}{*}{Descripción} & \multirow{2}{*}{ Empresa} &  \multirow{2}{*}{Cantidad} & Precio  & \multirow{2}{*}{Tiempo} & Precio \\
&  &  &  Unitario &  & Total\\
\hline
\hline
\endhead
\rule[0mm]{0mm}{4mm}
Instalador & Freedom & 2 & 878,63 &  2 meses & 3514.4\\
\hline
\rule[0mm]{0mm}{4mm}
Ingeniero & Freedom & 1 & 1.712,42 & 2 meses & 3424,84\\
\hline
Licencia & KNX & 1 & 1000 & - & 1000\\
\hline
Material & Varios & 1 & 12,759,56 & - & 12,459,56\\
\hline
\hline
\rule[0mm]{0mm}{4mm}
 & & & &TOTAL&20698.80 €\\
\hline 
\caption{Presupuesto}
\label{tab:tabla_presupuesto}
\end{longtable}
\end{center}